\documentclass[stu,a4paper,floatsintext]{apa7}

\usepackage{hyperref}
\usepackage{graphicx}
\usepackage{placeins}
\usepackage{lipsum}
\usepackage[american]{babel}
\usepackage{csquotes}
\usepackage[style=apa,sortcites=true,sorting=nyt,backend=biber]{biblatex}
\DeclareLanguageMapping{american}{american-apa}
\addbibresource{bibliography.bib}

\title{Underwater Object Recognition}
\shorttitle{Proposal Draft}

\author{Tommaso Cammelli\\Student ID: 23215488}
\authorsaffiliations{Faculy of Design and Creative Technologies, Auckland University of Technology}
\course{COMP 838: Deep Learning}
\duedate{5 April, 2024}
\professor{Dr. Wei Qi Yan}

% Abstract
\abstract{
    This project proposal investigates the efficacy of various deep learning
    architectures in the specialized field of underwater object recognition,
    a critical area of research with significant implications for marine
    biology, archaeological exploration, and autonomous underwater navigation.
    Recognizing the unique challenges posed by underwater environments such as
    low lighting, unclear images, and complex backgrounds.
    This project aims to perform a comparative analysis of various deep learning
    models and architecture like Convolutional Neural Networks,
    Residual Networks and Transformer models to find the most
    effective strategies for accurate and efficient object recognition in the
    underwater realm.

    By using Python as the primary programming language, the project will use
    popular deep learning libraries like TensorFlow and PyTorch, for model
    implementation and training.
    The methodology focuses on a comprehensive evaluation of model performance,
    assessing accuracy, precision, recall, and computational efficiency under
    varying underwater conditions.
    
    Expected outcomes include identifying optimal deep learning architectures
    that overcome the inherent challenges of underwater imaging.
}

% == DOCUMENT BEGIN ==
\begin{document}

\maketitle

% == IMPORTANT PART OF DOCUMENT ==
% 1- Introduction (No section for APA specs)
Underwater object recognition has emerged as a crucial field within marine
research, underpinning advancements in ecological monitoring, archaeological
exploration, and autonomous underwater vehicle navigation.
The unique challenges of underwater environments, such as limited visibility,
varying light conditions, and the presence of noise, demand robust and adaptive
solutions.
Deep learning offers a promising solution for addressing these challenges.

Recent years have witnessed the development of various deep learning
architectures, each with its strengths and limitations when applied to the
underwater domain.
Convolutional Neural Networks are well known for
using spatial hierarchies for effective feature extraction.
However, the introduction of architectures like Residual Networks (ResNets)
and Transformer models has opened new possibilities for enhancing recognition
accuracy and computational efficiency.
This project aims to conduct a comparative analysis of these architectures,
evaluating their performance in underwater object recognition tasks.

By analyzing the adaptability of these models to underwater conditions,
their scalability, and their efficiency in recognizing a diverse range of
objects, this project intends to identify optimal strategies
for deep learning-based underwater object recognition.
The comparison will not only focus on quantitative performance metrics but also
consider factors such as model complexity,
training data requirements, and real-world applicability.

The objects to be detected in this study are underwater animal species,
including echinus, holothurian, scallop, and starfish.
These species have been selected because datasets containing images of
these animals are more prevalent and easy to find.

\subsection{Sections of this proposal}

The proposal is divided in the following sections:

\begin{APAitemize}
    \item \textbf{Literature Review} Detailed examination of previous studies and advancements in underwater object recognition.
    \item \textbf{Methodology} Comprehensive outline of the data collection, model selection, implementation, and evaluation criteria.
    \item \textbf{Concluding Remarks} Summarization of findings and implications for future research in underwater object recognition.
\end{APAitemize}


%% #-- useful citation commands --#
%% | \Textcite{}                  |
%% | \parencite{}                 |
%% #------------------------------#

% 2- Literature Review (Heart of this proposal)
\section{Literature Review}
Underwater object recognition has been subject of increasing interest in the
scientific community, the interest is driven by the increasing number of
potential application in areas such as marine biology, underwater archeology
and autonomous underwater navigation.

\subsection{Deep Learning in Underwater Object Recognition}
Deep learning has significantly advanced the field of computer vision,
surpassing the capabilities of traditional object detection methods that
relied on hand-crafted features for image classification.
Initial endeavors predominantly employed Convolutional Neural Networks
(CNNs), used for their proficiency in automatically learning
and adapting spatial hierarchies of features from
images \parencite{zhiqiangReviewObjectDetection2017}.
For underwater detection CNN were used successfully to recognize objects with
significant accuracy, managing to work against the challenges of low-quality
images \parencite{hanUnderwaterImageProcessing2020}

\subsection{Advancements and Architectural Innovations}
Following the success of CNNs, more sophisticated architectures
like Residual Networks (ResNets) have been introduced, allowing training on
deeper networks by addressing the vanishing gradient problem 
\parencite{heDeepResidualLearning2016}.
ResNets have demonstrated remarkable performance in general object recognition
tasks, yet their efficiency and adaptability in underwater conditions remain
an area of active research.

The advent of Transformer models, originally designed for natural
language processing tasks, has been adapted for computer
vision \parencite{hanSurveyVisionTransformer2023}.
These models, which rely on self-attention mechanisms, offer a new
approach to handling the spatial relationships in images,
potentially offering advantages in complex underwater scenes where context
and object relationships are important.

\subsection{Challenges in Underwater environments}
Recognizing objects underwater presents unique challenges not typically
encountered in terrestrial environments.
Factors such as variable lighting conditions, water turbidity, and the
presence of particulates can significantly impact the performance of deep
learning models \parencite{liUnderwaterImageEnhancement2020e}.
Studies have begun to explore the robustness of different architectures under
such conditions, emphasizing the need for models that can adapt to or correct
for these environmental distortions.

\subsection{Comparative Analyses in the Literature}
Comparative studies specifically addressing the performance of deep learning
architectures in underwater object recognition are sparse but growing.
These studies are crucial for understanding the practical limitations and
opportunities of applying these advanced computational models to underwater
scenarios.
For instance, a study by \Textcite{tengUnderwaterTargetRecognition2020}
compared the accuracy and computational efficiency of CNNs
and ResNets in identifying underwater mines,
highlighting the trade-offs between model complexity and recognition
performance.



\section{Methodology}

\subsection{Data Collection}

In this project publicly available underwater image datasets will be used,
such as the Large Scale Underwater Image Dataset (LSUI) 
\parencite{pengUshapeTransformerUnderwater2023}
\footnote{LSUI dataset is available at
\url{https://lintaopeng.github.io/_pages/UIE\%20Project\%20Page.html}}.
These datasets contain a diverse range of underwater scenes, including various
flora, fauna, and man-made objects, providing a comprehensive basis for testing
and comparison.

Given the inherent challenges of underwater imaging,
such as varying light conditions and turbidity, images will undergo
pre-processing to enhance quality and consistency.
Techniques like color correction, contrast adjustment,
and noise reduction will be applied to mitigate environmental effects on
image quality, like shown in \parencite{tengUnderwaterTargetRecognition2020}.

\subsection{Model Selection and Development}

\begin{APAitemize}
    \item \textbf{CNNs}: Initial experiments will focus on conventional
     CNN architectures, which have proven effective in basic underwater object
     detection tasks.
     These models will serve as a benchmark for comparing more advanced
     architectures. % Add citations

    \item \textbf{Residual Networks (ResNets)}: Given their ability to train
     deeper networks by mitigating the vanishing gradient problem,
     ResNets will be explored for their potential to improve recognition
     accuracy in complex underwater scenes. % Add citations

     \item \textbf{Transformers}: The study will also incorporate Transformer
     models, which utilize self-attention mechanisms, to examine their
     effectiveness in capturing the spatial relationships of objects in
     underwater images. % Add citations
\end{APAitemize}

\subsection{Implementation}

The project will employ Python as the primary programming language,
leveraging its extensive ecosystem and the ease of finding pre-implemented
models along with Jupyter notebook to help the visualization of data.
For model implementation and training, I will utilize deep learning
libraries such as TensorFlow and PyTorch.
Experiments will be conducted on a laptop equipped with an Nvidia GPU
to facilitate efficient model training and evaluation.

\subsection{Evaluation Criteria}

Models will be evaluated based on accuracy, precision, recall, and F1 score
to determine their effectiveness in correctly identifying and classifying
underwater objects.
Additionally, computational efficiency, measured in terms of training time
and inference speed, will be considered to assess the practicality
of deploying these models in real-world applications.

The performance of each architecture will be compared to establish
their relative strengths and weaknesses in underwater object recognition.
This analysis will also explore the impact of varying dataset complexities
and environmental conditions on model performance.

\section{Concluding Remarks}

This project proposal outlines a comprehensive approach to exploring the
potential of various deep learning architectures in the challenging domain of
underwater object recognition.
By using Python with the capabilities of TensorFlow, PyTorch,
and the interactive environment of Jupyter Notebooks, this study aims to not
only compare the efficacy of established models such as CNNs and ResNets but
also investigate the emerging potential of Transformer models in this context.

This project could potentially contribute to the field of marine biology,
archaeological exploration and autonomous underwater navigation by identifying
optimal deep learning strategies that can help with the unique challenges
posed by the underwater environments.

\printbibliography

%\appendix - <if needed>
\end{document}



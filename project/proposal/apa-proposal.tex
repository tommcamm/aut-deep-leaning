\documentclass[stu,a4paper,floatsintext]{apa7}

\usepackage{amsmath}
\usepackage{array}
\usepackage{multirow}
\usepackage{hyperref}
\usepackage{graphicx}
\usepackage{placeins}
\usepackage{lipsum}
\usepackage[american]{babel}
\usepackage{csquotes}
\usepackage[style=apa,sortcites=true,sorting=nyt,backend=biber]{biblatex}

\DeclareLanguageMapping{american}{american-apa}
\addbibresource{bibliography.bib}

\title{Underwater Object Recognition}

\author{Tommaso Cammelli\\Student ID: 23215488}
\authorsaffiliations{Faculty of Design and Creative Technologies, Auckland University of Technology}
\course{COMP 838: Deep Learning}
\duedate{10 June, 2024}
\professor{Dr. Wei Qi Yan}

% Abstract
\abstract{
    This project proposal aims to investigate the efficacy of various deep learning
    architectures in the specialized field of underwater object recognition,
    which has significant implications for marine biology, archaeological exploration, and autonomous underwater navigation.
    Underwater environments pose unique challenges such as low lighting, unclear images, and complex backgrounds,
    necessitating robust solutions.

    This project will conduct a comparative analysis of deep learning models
    including Convolutional Neural Networks (CNNs), Residual Networks (ResNets),
    and Transformer models to determine the most effective strategies for accurate
    and efficient object recognition in underwater scenarios.

    Using Python and by using popular deep learning libraries such as
    TensorFlow and PyTorch, the project will implement and train these models.
    The methodology involves a comprehensive evaluation of model performance,
    focusing on metrics like accuracy, precision, recall, F1 score,
    and computational efficiency under various underwater conditions.

    The expected outcomes include identifying the optimal deep learning
    architectures that can effectively address the challenges of underwater imaging.
}


% = Custom commands

\newcommand\MyBox[2]{
  \fbox{\lower0.75cm
    \vbox to 1.7cm{\vfil
      \hbox to 1.7cm{\hfil\parbox{1.4cm}{#1\\#2}\hfil}
      \vfil}%
  }%
}


% == DOCUMENT BEGIN ==
\begin{document}

\maketitle

% == IMPORTANT PART OF DOCUMENT ==
Let's begin with an introduction...

%% #-- useful citation commands --#
%% | \Textcite{}                  |
%% | \parencite{}                 |
%% #------------------------------#

% 2- Literature Review (Heart of this proposal)
\section{Literature Review}
Underwater object recognition has been subject of increasing interest in the
scientific community, driven by the increasing number of
potential application in areas such as marine biology, underwater archaeology,
and autonomous underwater navigation.

\subsection{Deep Learning in Underwater Object Recognition}
Deep learning has significantly advanced the field of computer vision,
surpassing the capabilities of traditional object detection methods that
relied on hand-crafted features for image classification.
Initial endeavors predominantly employed Convolutional Neural Networks
(CNNs), used for their proficiency in automatically learning
and adapting spatial hierarchies of features from
images \parencite{zhiqiangReviewObjectDetection2017}.
For underwater detection, CNNs were used successfully to recognize objects with
significant accuracy, managing to work against the challenges of low-quality
images \parencite{hanUnderwaterImageProcessing2020}.
In particular, the YOLO (You Only Look Once) series, including YOLOv4,
has demonstrated notable effectiveness.
\Textcite{rosliUnderwaterAnimalDetection2021} showed that YOLOv4 achieved a remarkable 97.96\%
mean average precision (mAP) and 46.6 frames per second (FPS) in underwater detection tasks,
emphasizing its capability in handling varying visibility and low-light environments.

\subsection{Advancements and Architectural Innovations}
Following the success of CNNs, more sophisticated architectures
like Residual Networks (ResNets) have been introduced, allowing training on
deeper networks by addressing the vanishing gradient problem 
\parencite{heDeepResidualLearning2016}.
ResNets have demonstrated remarkable performance in general object recognition
tasks, yet their efficiency and adaptability in underwater conditions remain
an area of active research.

The advent of Transformer models, originally designed for natural
language processing tasks, has been adapted for computer
vision \parencite{hanSurveyVisionTransformer2023}.
These models, which rely on self-attention mechanisms, offer a new
approach to handling the spatial relationships in images,
potentially offering advantages in complex underwater scenes where context
and object relationships are important.

\subsection{Challenges in Underwater Environments}
Recognizing objects underwater presents unique challenges not typically
encountered in terrestrial environments.
Factors such as variable lighting conditions, water turbidity, and the
presence of particulates can significantly impact the performance of deep
learning models \parencite{liUnderwaterImageEnhancement2020}.
Studies have begun to explore the robustness of different architectures under
such conditions, emphasizing the need for models that can adapt to or correct
for these environmental distortions.

\subsection{Comparative Analyses in the Literature}
Comparative studies specifically addressing the performance of deep learning
architectures in underwater object recognition are sparse but growing.
These studies are crucial for understanding the practical limitations and
opportunities of applying these advanced computational models to underwater
scenarios.
For instance, a study by \Textcite{tengUnderwaterTargetRecognition2020}
compared the accuracy and computational efficiency of CNNs
and ResNets in identifying underwater mines,
highlighting the trade-offs between model complexity and recognition
performance.

Additionally, \Textcite{pedersenDetectionMarineAnimals2019} introduced the Brackish Dataset,
a new publicly available underwater dataset containing annotated image sequences of fish,
crabs, and starfish captured in brackish water with varying visibility.
They evaluated the performance of YOLOv2 and YOLOv3 on this dataset,
establishing a baseline for future studies. 
This contribution is significant as it provides a unique dataset and highlights
the importance of robust data for training and evaluating marine object recognition models.

\subsection{Conclusion}
The reviewed literature shows the rapid advancements and diverse applications
of deep learning in underwater object recognition.
While Convolutional Neural Networks have laid a robust foundation,
the emergence of architectures like Residual Networks and Transformer models offers
promising avenues for enhancing recognition accuracy and efficiency.
However, the unique challenges posed by underwater environments necessitate further
research and the development of adaptive models.
The introduction of comprehensive datasets,
such as the Brackish Dataset by \Textcite{pedersenDetectionMarineAnimals2019}, plays
an important role, providing the necessary data for training and evaluation.

\section{Methodology}

\begin{figure}[tb]
    \begin{minipage}{0.48\textwidth}
      \centering
      \includegraphics[width=.7\linewidth]{figures/pre-processing.png}
      \caption{Input underwater Image}
      \label{Fig:PreProcessing}
    \end{minipage}\hfill
    \begin{minipage}{0.48\textwidth}
      \centering
      \includegraphics[width=.7\linewidth]{figures/post-processing.png}
      \caption{Enhanced Image with recognized objects}
      \label{Fig:PostProcessing}
    \end{minipage}
\end{figure}

\subsection{Data Collection}

In this project publicly available underwater image datasets will be used,
such as the Underwater Object Detection Dataset (UODD) 
\parencite{jiangUnderwaterSpeciesDetection2021}
\footnote{UODD dataset is available at
\url{https://github.com/LehiChiang/Underwater-object-detection-dataset}}.
These datasets contain a diverse range of underwater scenes, including various
flora, fauna, and man-made objects, providing a comprehensive basis for testing
and comparison, along with annotations labeled in MS COCO format.

Given the inherent challenges of underwater imaging,
such as varying light conditions and turbidity, images will undergo
pre-processing to enhance quality and consistency.
Techniques like color correction, contrast adjustment,
and noise reduction will be applied to mitigate environmental effects on
image quality, like shown in \parencite{tengUnderwaterTargetRecognition2020}
and \parencite{pengUshapeTransformerUnderwater2023}.

\subsection{Model Selection and Development}

\begin{APAitemize}
    \item \textbf{CNNs}: Initial experiments will focus on conventional
     CNN architectures, which have proven effective in basic underwater object
     detection tasks.
     These models will serve as a benchmark for comparing more advanced
     architectures. % Add citations

    \item \textbf{Residual Networks (ResNets)}: Given their ability to train
     deeper networks by mitigating the vanishing gradient problem,
     ResNets will be explored for their potential to improve recognition
     accuracy in complex underwater scenes. % Add citations

     \item \textbf{Transformers}: The study will also incorporate Transformer
     models, which utilize self-attention mechanisms, to examine their
     effectiveness in capturing the spatial relationships of objects in
     underwater images. % Add citations
\end{APAitemize}

\subsection{Implementation}

The project will employ Python as the primary programming language,
leveraging its extensive ecosystem and the ease of finding pre-implemented
models along with Jupyter notebook to help the visualization of data.
For model implementation and training, I will utilize deep learning
libraries such as TensorFlow and PyTorch.
Experiments will be conducted on a laptop equipped with an Nvidia GPU
to facilitate efficient model training and evaluation.

\subsection{Evaluation Criteria}

The evaluation of the various models will use common indicators like accuracy,
precision, recall and F1 score to determine the effectiveness of each model
in correctly identifying and classifying underwater objects.
Additionally, computational efficiency, measured in terms of training time
and inference speed, will be considered to assess the practicality
of deploying these models in real-world applications.

The performance of each architecture will be compared to establish
their relative strengths and weaknesses in underwater object recognition.
This analysis will also explore the impact of varying dataset complexities
and environmental conditions on model performance.

\FloatBarrier

\section{Concluding Remarks}

This project proposal outlines a comprehensive approach to exploring the
potential of various deep learning architectures in the challenging domain of
underwater object recognition. By leveraging Python and powerful deep learning
libraries such as TensorFlow and PyTorch, alongside the interactive capabilities of
Jupyter Notebooks, this study aims to compare the efficacy of established models such as CNNs and ResNets, as well as investigate the emerging potential of Transformer models in this context.

A significant focus of this project will be on the pre-processing of images,
as the quality and clarity of underwater images can significantly impact
the performance of deep learning models.
Techniques such as image enhancement, noise reduction, and color correction are
expected to play a crucial role in improving the accuracy and robustness
of object recognition.

The insights gained from this project will contribute to a deeper understanding
of the strengths and limitations of different deep learning approaches in underwater
environments.

\printbibliography

\clearpage

\appendix

\section{Responses to Reviewers Comments on the Project Proposal}

\vspace{1em}

\subsection{Reviewer 1}

\textbf{Question 1.1:} Consider may adding more detail to the introduction on the types of objects you will be detecting underwater. It is mentioned in the methodology but I believe that it would be good to know that in the introduction as well.\\
\textbf{Answer 1.1:} Updated the introduction adding type of objects that will be detected (Animal Species).

\vspace{1em}

\textbf{Question 1.2:} It would be helpful if the literature review includes a conclusion. Having a conclusion will help get summarised results and some key findings of the literature review.\\
\textbf{Answer 1.2:} Updated the literature review section adding a conclusion.

\vspace{1em}

\subsection{Reviewer 2}

\textbf{Question 2.1:} ..\\
\textbf{Answer 2.1:} ..

\vspace{1em}

\textbf{Question 2.2:} ..\\
\textbf{Answer 2.2:} ..

\vspace{1em}

\subsection{Reviewer 3}

\textbf{Question 3.1:} ..\\
\textbf{Answer 3.1:} ..

\vspace{1em}

\textbf{Question 3.2:} ..\\
\textbf{Answer 3.2:} ..

\end{document}
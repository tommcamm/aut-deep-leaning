% 1- Introduction (No section for APA specs)
Underwater object recognition has emerged as a crucial field within marine
research, underpinning advancements in ecological monitoring, archaeological
exploration, and autonomous underwater vehicle navigation.
The unique challenges of underwater environments, such as limited visibility,
varying light conditions, and the presence of noise, demand robust and adaptive
solutions.
Deep learning offered a promising solution for addressing these challenges.

Recent years have witnessed the development of various deep learning
architectures, each with its strengths and limitations when applied to the
underwater domain.
Convolutional Neural Networks are well known for
using spatial hierarchies for effective feature extraction.
However, the introduction of architectures like Residual Networks (ResNets)
and Transformer models have opened new possibilities for enhancing recognition
accuracy and computational efficiency.
This project aims to conduct a comparative analysis of these architectures,
evaluating their performance in underwater object recognition tasks.

By analyzing the adaptability of these models to underwater conditions,
their scalability, and their efficiency in recognizing a diverse range of
objects, this project intends to identify optimal strategies
for deep learning-based underwater object recognition.
The comparison will not only focus on quantitative performance metrics but also
consider factors such as model complexity,
training data requirements, and real-world applicability.

\vspace{10pt}

\subsection{Sections of this proposal}

The proposal is divided in the following sections:

\begin{APAitemize}
    \item \textbf{Literature Review} ...
    \item \textbf{Methodology} ... 
    \item \textbf{Concluding Remarks} ...
\end{APAitemize}
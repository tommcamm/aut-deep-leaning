% 1- Introduction (No section for APA specs)
Underwater object recognition has emerged as a crucial field within marine
research, underpinning advancements in ecological monitoring, archaeological
exploration, and autonomous underwater vehicle navigation.
The unique challenges of underwater environments, such as limited visibility,
varying light conditions, and the presence of noise, demand robust and adaptive
solutions.
Deep learning, with its capacity to learn hierarchical representations,
offers a promising avenue for addressing these challenges.

Recent years have witnessed the development of various deep learning
architectures, each with its strengths and limitations when applied to the
underwater domain.
Convolutional Neural Networks (CNNs) have been at the forefront,
leveraging spatial hierarchies for effective feature extraction.
However, the introduction of architectures like Residual Networks (ResNets)
and Transformer models has opened new possibilities for enhancing recognition
accuracy and computational efficiency.
This project aims to conduct a comparative analysis of these architectures,
evaluating their performance in underwater object recognition tasks.

By analyzing the adaptability of these models to underwater conditions,
their scalability, and their efficiency in recognizing a diverse range of
objects, this research intends to identify optimal strategies
for deep learning-based underwater object recognition.
The comparison will not only focus on quantitative performance metrics but also
consider factors such as model complexity,
training data requirements, and real-world applicability.
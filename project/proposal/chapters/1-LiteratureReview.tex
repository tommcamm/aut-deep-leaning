%% #-- useful citation commands --#
%% | \Textcite{}                  |
%% | \parencite{}                 |
%% #------------------------------#

% 2- Literature Review (Heart of this proposal)
\section{Literature Review}
Underwater object recognition has been subject of increasing interest in the
scientific community, driven by the increasing number of
potential application in areas such as marine biology, underwater archaeology,
and autonomous underwater navigation.

\subsection{Deep Learning in Underwater Object Recognition}
Deep learning has significantly advanced the field of computer vision,
surpassing the capabilities of traditional object detection methods that
relied on hand-crafted features for image classification.
Initial endeavors predominantly employed Convolutional Neural Networks
(CNNs), used for their proficiency in automatically learning
and adapting spatial hierarchies of features from
images \parencite{zhiqiangReviewObjectDetection2017}.
For underwater detection, CNNs were used successfully to recognize objects with
significant accuracy, managing to work against the challenges of low-quality
images \parencite{hanUnderwaterImageProcessing2020}.
In particular, the YOLO (You Only Look Once) series, including YOLOv4,
has demonstrated notable effectiveness.
\Textcite{rosliUnderwaterAnimalDetection2021} showed that YOLOv4 achieved a remarkable 97.96\%
mean average precision (mAP) and 46.6 frames per second (FPS) in underwater detection tasks,
emphasizing its capability in handling varying visibility and low-light environments.

\subsection{Advancements and Architectural Innovations}
Following the success of CNNs, more sophisticated architectures
like Residual Networks (ResNets) have been introduced, allowing training on
deeper networks by addressing the vanishing gradient problem 
\parencite{heDeepResidualLearning2016}.
ResNets have demonstrated remarkable performance in general object recognition
tasks, yet their efficiency and adaptability in underwater conditions remain
an area of active research.

The advent of Transformer models, originally designed for natural
language processing tasks, has been adapted for computer
vision \parencite{hanSurveyVisionTransformer2023}.
These models, which rely on self-attention mechanisms, offer a new
approach to handling the spatial relationships in images,
potentially offering advantages in complex underwater scenes where context
and object relationships are important.

\subsection{Challenges in Underwater Environments}
Recognizing objects underwater presents unique challenges not typically
encountered in terrestrial environments.
Factors such as variable lighting conditions, water turbidity, and the
presence of particulates can significantly impact the performance of deep
learning models \parencite{liUnderwaterImageEnhancement2020}.
Studies have begun to explore the robustness of different architectures under
such conditions, emphasizing the need for models that can adapt to or correct
for these environmental distortions.

\subsection{Comparative Analyses in the Literature}
Comparative studies specifically addressing the performance of deep learning
architectures in underwater object recognition are sparse but growing.
These studies are crucial for understanding the practical limitations and
opportunities of applying these advanced computational models to underwater
scenarios.
For instance, a study by \Textcite{tengUnderwaterTargetRecognition2020}
compared the accuracy and computational efficiency of CNNs
and ResNets in identifying underwater mines,
highlighting the trade-offs between model complexity and recognition
performance.

Additionally, \Textcite{pedersenDetectionMarineAnimals2019} introduced the Brackish Dataset,
a new publicly available underwater dataset containing annotated image sequences of fish,
crabs, and starfish captured in brackish water with varying visibility.
They evaluated the performance of YOLOv2 and YOLOv3 on this dataset,
establishing a baseline for future studies. 
This contribution is significant as it provides a unique dataset and highlights
the importance of robust data for training and evaluating marine object recognition models.

\subsection{Conclusion}
The reviewed literature shows the rapid advancements and diverse applications
of deep learning in underwater object recognition.
While Convolutional Neural Networks have laid a robust foundation,
the emergence of architectures like Residual Networks and Transformer models offers
promising avenues for enhancing recognition accuracy and efficiency.
However, the unique challenges posed by underwater environments necessitate further
research and the development of adaptive models.
The introduction of comprehensive datasets,
such as the Brackish Dataset by \Textcite{pedersenDetectionMarineAnimals2019}, plays
an important role, providing the necessary data for training and evaluation.
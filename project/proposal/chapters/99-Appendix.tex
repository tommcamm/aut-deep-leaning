\appendix

\section{Responses to Reviewers Comments on the Project Proposal}

\vspace{1em}

\subsection{Reviewer 1}

\textbf{Question 1.1:} Consider may adding more detail to the introduction on the types of objects you will be detecting underwater. It is mentioned in the methodology but I believe that it would be good to know that in the introduction as well.\\
\textbf{Answer 1.1:} Updated the introduction adding type of objects that will be detected (Animal Species).

\vspace{1em}

\textbf{Question 1.2:} It would be helpful if the literature review includes a conclusion. Having a conclusion will help get summarised results and some key findings of the literature review.\\
\textbf{Answer 1.2:} Updated the literature review section adding a conclusion.

\vspace{1em}

\subsection{Reviewer 2}

\textbf{Question 2.1:} You could add some details about the dataset you are using, and some illustrations so we could understand easier why you are using this dataset.\\
\textbf{Answer 2.1:} Added details about the used dataset in both introduction and methodology.

\vspace{1em}

\textbf{Question 2.2:} You should be more understandable in your methodology part. You could add some mathematical explanations about the concepts you talked about\\
\textbf{Answer 2.2:} Added more details about the models and concepts used in the methodology section along with formulas.

\vspace{1em}

\pagebreak

\subsection{Reviewer 3}

\textbf{Question 3.1:} Explain the structure of the dataset and the model. Is it zoomed in on a particular element and then labelled? For the dataset, you should probably explain in detail the different features you are using. For the model, just make a scheme of your project's architecture.\\
\textbf{Answer 3.1:} Updated details about the dataset and the model in the methodology section.

\vspace{1em}

\textbf{Question 3.2:} Instead of focusing on the confusion matrix equations themselves, talk about their practical application within your proposal. I would use the equations to detail some parts of the model you are working on. The transformative models, for example seem really interesting to study mathematically.\\
\textbf{Answer 3.2:} Updated validation criteria and added more details about each model.

\vspace{1em}

\textbf{Question 3.3:} Downsizing the repetition of the concepts is not an easy task. However, your project might benefit from finding alternative ways to convey this information. In the case mentioned above, I would maybe detail one specific use-case like archaeology and explain how it can benefit from what you are studying.\\
\textbf{Answer 3.3:} Restructured text to limit repetition in all sections.